\documentclass[12pt, a4paper]{report}
\usepackage[utf8]{inputenc}
\usepackage{hyperref} 
\title{Machine Learning Course Project}

\begin{document}
    %\maketitle
    %\tableofcontents
    \section{Problem formulation}
    \subsection{Application of ML Problem}
    The current winter has shown that slippery weather might come as a surprise.
    However, when slippery conditions are expected during the day, a service called \href{https://liukastumisvaroitus.fi/en/}{Liukastumisvaroitus} (Slipping warning) sends subscribers a text-message.

    According to the website, the slippery condition is identified by humans.
    Though one could think that these dangerous weather conditions can be predicted without 
    human knowledge, but to what level of accuracy? This makes it a great target for testing machine learning as an application.
    
    Ideally the machine learning application could predict, given the current weather conditions, whether the slippery warning would be raised.

    The data point is going to be a daily observation of weather data, with the additional
    slippery warning parameter. In other words, a single data point represents the weather conditions of a day. Data includes all data points (daily observations) from around 
    November 2013, as the earliest records of slipping warnings are from then.

    Concluding the parameters, 
    \begin{itemize}
      \item Potential features for the application could be \textit{Precipitation amount}, \textit{Air temperature} and \textit{Snow depth}. All of these properties are numerical and are easily measureable.
      \item The label of the application is going to be whether the \textit{slippery warning} would be raised, with values 0/1.
    \end{itemize}
    

    \subsection{Data sources}
    The slipping warning service offers an \href{https://liukastumisvaroitus-api.beze.io/api/v1/warnings/}{API} for historical data analysis. Some 600 warnings have been issued in total  since November 2013.
    %The slipping warning can be seen to be given at any point during the day, though most usually in the night hours.
    As the slipping warning service data only consists of a timestamp and the city issued, training the machine learning algorithm to account for the current weather conditions needs more data to work with.

    Thus, historical weather data from the \href{https://en.ilmatieteenlaitos.fi/download-observations}{Finnish Metheorological Institute}
    is combined. Although FMI offers hourly historical data, in this project, the plan is to use the daily aggregated weather recordings for simplicity. In total, there are some 3000 daily weather reports since November 2013
    
    Combined together, these data sources will be used to train and validate the machine learning algorithm.
            
\end{document}
